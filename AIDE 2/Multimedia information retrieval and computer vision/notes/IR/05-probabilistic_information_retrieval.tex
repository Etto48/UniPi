\subsection{Probabilistic information retrieval}
\label{sec:probabilistic_information_retrieval}

Thus far the queries have all been boolean queries.
Most of times boolean queries return 0 or too many results.
We want to return results ordered by usefulness.
We need probability to calculate the usefulness of a document.

We have a collection of documents.
A user issues a query.
A list of $k$ documents must be returned.
In what order do we present these documents?
How can we compute the best order?
We rank the documents in the collection according to the probability
that the document is relevant to the query.
$P(\text{relevant}|\text{document},\text{query})$.

A user has an information need.
This user encodes this need as a textual query.
A user expresses a relevance judgment on a given document for a given
information need expressed as a query.
A relevance judgment is associated with a document-query pair.
Those judgments are hidden and not explicitly observable.
A relevance is a document property depending uniquely on the 
information need. It's independent from other documents.
A relevance judgment is a binary variable.

The \textbf{probabilistic ranking principle} states that the best way 
(with respect to effectiveness) to rank a set of documents
is by decreasing probability of relevance.

Relevance is a binary random variable $R$.
The query is a random variable $Q$ with generic value $q$.
Document $D$ is a random variable with generic value $d$.
We denote $P(R=r|Q=q,R=r)$ with $P(r|d,q)$ for brevity.
The system returns $k$ documents in decreasing order of $P(r|d,q)$.
Those documents are called $d_1,\dots,d_k$.
Recall $R_k$ is a random variable denoting the number of 
relevant documents retrieved for a given query so it's in $\{0,\dots,k\}$.
The overall effectiveness is measured by expected number
of relevant documents retrieved $E[R_k]$.

With those assumptions the maximum value we can achieve of $E[R_k]$
is obtained with the \textbf{probabilistic ranking principle}.
\[
    E[R_k] = r\sum_{i=1}^{k}P(r|d_i,q)+\bar{r}\sum_{i=1}^{k}P(\bar{r}|d_i,q) = 
    \sum_{i=1}^{k}P(r|d_i,q)
\]

The easiest way to prove this is by contradiction.
Assume that the best way to rank documents is not by decreasing
probability of relevance.
Then there exists a ranking that is better than the one given by
the probabilistic ranking principle.
So it means that there is at least a document $d_l$ with $l<k$such that
$P(r|d_l,q)$ is less than $P(r|d_{\bar k},q)$ with $\bar k>k$.
If we swap $d_l$ with $d_k$ we get a better $E[R_k]$.

It's important to note that in the assumptions we said that
the probability does not depend on:
\begin{itemize}
    \item other users.
    \item other information needs.
    \item other queries of the same user.
    \item other documents returned for the same query.
\end{itemize}

Maximizing $E[R_k]$ is equivalent to maximizing precision and recall
at cutoff $k$.

If the probability depends on the user, the PRP is not necessarily valid.

We want to estimate $P(r|d,q)$. We can use Bayes' theorem.
\[
    P(r|d,q) = \frac{P(d|r,q)P(r)}{P(d)}
\]
$P(d|r,q)$ is the probability of observing document $d$ given that
it is relevant to query $q$.
$P(r)$ is the prior probability retrieving a relevant document at random.
$P(d)$ is the probability of observing document $d$.
\begin{itemize}
    \item Estimate how each term contributes to the relevance.
    \item We assume that terms not appearing in the document have 0 contribution.
    \item How do term occurrences influence the relevance of a document.
    \item How do document lengths influence the relevance of a document.
    \item Combine term contributions together to dinf the document relevance.
    \item Order documents by decreasing relevance.
\end{itemize}

\subsubsection{Binary Independence Model}
\label{sec:binary_independence_model}

Every document $d$ is represented as a binary vector of terms.
This vector is $x_d = (x_{d1},\dots,x_{dn})$ where $x_{di}$ is 1 if
term $i$ appears in document $d$ and 0 otherwise.

This model assumes that each term is independent of the others.

We are not going to compute probabilities because it's too difficult.
We will compute a score that will be ordered in the same way as the
probabilities.

We will use \textbf{rank-preserving functions}. These functions
are functions that preserve the ordering of the inputs.
(Not that different from monotonic functions).

Given an event $E$ the odds of $E$ are defined as:
\[
    O(E) = \frac{P(E)}{P(\bar{E})} = \frac{P(E)}{1-P(E)}
\]
This is a \textbf{rank-preserving function}.

The odds of being relevant given a document and a query are:
\[
    O(r|d,q) = \frac{P(r|d,q)}{P(\bar{r}|d,q)} = 
        \frac{P(r|q)}{P(\bar{r}|q)}\frac{P(d|r,q)}{P(d,\bar{r},q)}
\]
The first fraction $\frac{P(r|q)}{P(\bar{r}|q)}$ is a constant and 
does not influence the ranking.
We use the independence assumption:
(We use $p_i=P(x_i=1|r,q)$ and $r_i=P(x_i=1|\bar{r},q)$ 
then $P(x_i=0|r,q)=1-p_i$ and $P(x_i=0|\bar{r},q)=1-r_i$
$x_i$ is the $i$-th term in the document).
\[
    \frac{P(d|r,q)}{P(d,\bar{r},q)} =
        \prod_{x_i=1}\frac{p_i}{r_i}\prod_{x_i=0}\frac{1-p_i}{1-r_i}
\]
We apply the hypothesis:
\[
    O(r|d,q)\propto\prod_{x_i=1,q_i=1}\frac{p_i}{r_i}
        \prod_{x_i=0,q_i=1}\frac{1-p_i}{1-r_i}
\]
We multiply and divide by $prod_{x_i=1,q_i=1}\frac{1-p_i}{1-r_i}$.
\[
    O(r|d,q)\propto\prod_{x_i=1,q_i=1}\frac{p_i(1-r_i)}{r_i(1-p_i)}
        \prod_{q_i=1}\frac{1-p_i}{1-r_i}
\]
For a given query $\prod_{q_i=1}\frac{1-p_i}{1-r_i}$ is constant.
So
\[
    O(r|d,q)\propto\prod_{x_i=1,q_i=1}\frac{p_i(1-r_i)}{r_i(1-p_i)}
\]
The product is carried out only for terms that appear both in the query
and in the document.
The output of this function is really close to 0 so we use the
logarithm to make it more readable.
Of course $log$ is a \textbf{rank-preserving function}.
\[
    \text{RSV}=log\prod_{x_i=1,q_i=1}\frac{p_i(1-r_i)}{r_i(1-p_i)} =
        \sum_{x_i=1,q_i=1}log\frac{p_i(1-r_i)}{r_i(1-p_i)} =
        \sum_{x_i=1,q_i=1} c_i
\]
Where $c_i=log\frac{p_i(1-r_i)}{r_i(1-p_i)}$.
We only need to compute $p_i$ and $r_i$.
We need to find a way to estimate $p_i$ and $r_i$.
Let's assume that we have a random sample of our collection with complete
relevance judgments.
Let
\begin{enumerate}
    \item $N$ be the number of documents in the collection.
    \item $n_i$ be the number of documents in the sample containing term $i$.
    \item $R$ be the number of relevant documents in the collection.
    \item $r_i$ be the number of relevant documents in the sample containing term $i$.
\end{enumerate}

We add constants to make the computation more stable and in the end we get:
\[
    p_i = \frac{r_i+0.5}{R+1}\\
    r_i = \frac{n_i-r_i+0.5}{N-R+1}
\]
Keep in mind that $p_i$ is the probability of a term being in a relevant
document and $r_i$ is the probability of a term being in a non-relevant
document.

The \textbf{Robertson/Sparck-Jones model} tells us that
\[
    c_i^{BIM} = w_i^{RSJ} = 
        log\frac{(r_i+0.5)(N-R-n_i+r_i+0.5)}{(n_i-r_i+0.5)(R-r_i+0.5)}
\]
If we have no clue we can set $r_i=0$ and $p_i=0.5$.
So the simplest way to compute the relevance of a document is using
$log\frac{N}{n_i}$ and this is the \textbf{best match 1} aka \textbf{BM1} 
aka \textbf{IDF}.

\subsubsection{Generative models for documents}
\label{sec:generative_models_for_documents}

In a generative model we assume that the document is generated by
randomly and independently selecting terms from a vocabulary using
a multinomial distribution.
Assuming that every term has the same probability of being selected
is not realistic so we give each term a probability of being selected
proportional to its frequency in the collection.
We are only interested in the occurrences for each term, not in the order.
We assume that the $tf$ follows a Poisson distribution.
\[
    f(k|\mu) = P(X=k|\mu) = \frac{\mu^ke^{-\mu}}{k!}
\]

The \textbf{1-Poisson model} works well with general words, but underperforms
with topic-specific words.
The \textbf{2-Poisson model} is better for topic-specific words.
In the \textbf{2-Poisson model} we have a hidden binary random variable
$E_i$ that is called eliteness for each document-term pair $i$.
Its value is 1 if the document is about the topic represented by the term
and 0 otherwise.
The event $E_i=1$ is denoted as $e_i$ and the event $E_i=0$ is denoted as
$\bar{e}_i$.
So we have 
\[
    e_i^{elite}(tf) = log\frac{P(tf_i|r)P(0,|\bar{r})}{P(0|r)P(tf_i|\bar{r})}
\]
We can estimate $P(tf_i|r)$ like this:
\[
    P(tf_i|r) = \pi\frac{\lambda^k}{k!}e^{-\lambda}+(1-\pi)\frac{\mu^k}{k!}e^{-\mu}
\]
Where $\pi = P(e_i|r)$.
Unfortunately we don't know $\pi$, $\lambda$ and $\mu$.

How can we estimate $c_i^{elite}$?
We can approximate it with 
$c_i^{BM15}(tf_i)\approxeq log\frac{N}{n_i}\frac{tf_i}{k_1+tf_i}$.
This is the \textbf{best match 15} aka \textbf{BM15} which is similar to
\textbf{TFIDF} but with bounded term scores.

Now we need to take in account the length of the document.
We can have a long document because of its verbosity or because it's
spanning multiple topics.
We have cases in which longer documents provide an overinflated term frequency
and cases in which the term frequency is uneffected by the size of the document.
In real cases it's always an in-between. We have to find a normalizing factor.
We can compute the average document length $\text{avdl}=\frac{1}{N}\sum_{d=1}^{N}dl_d$
where $dl_d$ is the length of document $d$.
We can then recompute the term frequency as $tf'_i(d_j)=tf_i(d_j)\frac{avdl}{dl_j}$.
Using this in the \textbf{BM15} model we get the \textbf{BM11} model.

What if we had a document that is really long but not verbose?
We can use a partial normalization, interpolating between $tf$ and $tf'$ using a 
parameter $b$. When $b=0$ we have no normalization, when $b=1$ we have full normalization.

If we chose $b\approxeq0.75$ and $k_1\in[1.2,2]$ in this way we obtain the \textbf{Okapi BM25} model.

We have everything we need in the inverted index and the document index.
The only hyperparameters left are $k_1$ and $b$.
\[
    RSV^{BM25}(q,d) = 
        \sum_{t_i\in q}\frac{tf_i(d)}{k_1((1-b)+b\frac{dl(d)}{avdl})+tf_i(d)}log\frac{N}{n_i}
\]

When we have multiple fields due to structure in the document we can use the \textbf{BM25F} model.
This model calculates a weighted sum of the $tf$ for each field for a term.
The weights are \textbf{really} difficult to estimate.