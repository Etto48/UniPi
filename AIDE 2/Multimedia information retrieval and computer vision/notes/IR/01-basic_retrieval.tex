\subsection{Basic Retrieval}
\label{sec:basic_retrieval}

An IR system resembles and has evolved from a library.
Now they are Google, DuckDuckGo, Bing, etc.
Whenever we have a "search bar" we are using an IR system.
Also image, audio, legal and medical retrieval systems exist.

Imagine searching for a word in a book, it would be
slow enough with $O(100)$ pages, imagine with $O(1B)$ pages.
Linear search is not feasible, that's why we need IR systems.

The procedure of retrieving relevant information from data sources
which were not originally designed for information access is called
\textbf{Information Retrieval}.

An IR system is assessed by its efficiency (time to retrieve) and
effectiveness (relevance of the retrieved information).

The kind of information that IR systems handle can be any kind
of text (txt, pdf, word files, forum posts, etc.), images, audio,
video, etc.

Why can't we use a database system for IR? Because databases are
designed for structured data specifically for relations and entities.
A well defined structure and semantics is required for databases.
If some values are Null we encounter problems like performance 
issues etc.
In text files we don't have a well defined structure, we have
unstructured data.

An IR system aims to rank documents by relevance to a query.
The relevance is a score assigned to a document basend on its
content related to the query.
Often exact matches are not enough to determine relevance
because they are too strict. We need to consider the properties
of natural language.

We have to deal with "fuzzyness" in IR systems.
Most of the stuff in IR is an heuristic not an exact algorithm.

What makes a document relevant?
\begin{itemize}
    \item it contains the query terms?
    \item it contains all of the query terms?
    \item it contains the query terms many times?
    \item it contains the query terms many times in a short document?
    \item it contains the query terms in close proximity?
    \item \dots
\end{itemize}

In the typical scenario of an IR we have a user that issues a query,
some kind of question. The system provides an answer and then the user
can provide an assessment of the answer. The aim of a search engineer
is to make this process automatic.
