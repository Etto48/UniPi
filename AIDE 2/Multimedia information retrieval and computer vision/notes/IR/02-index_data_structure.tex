\subsection{Index Data Structure}
\label{sec:index_data_structure}

To find a query result in a big corpora of documents is
no easy task, with grep we could achieve $O(n)$ time complexity
but this does not scale well.

\textbf{Term-document incidence matrix} is a matrix where rows are
terms and columns are documents. The matrix is filled with
1s and 0s, 1 if the term is in the document, 0 otherwise.
\textbf{Incidence vectors} are the rows of the matrix.
to find an AND query we can simply multiply (bitwise AND) 
the vectors and find the result. To find a NOT query we can
negate the vector and then multiply.
The problem with this approach is the size of the matrix,
it grows with the number of terms and documents $O(n*m)$.
The \textbf{term-document incidence matrix} is extremely sparse
(a lot of 0s) so we can only store the positions of the 1s.
Doing this we get an \textbf{inverted index}.

\textbf{Inverted indexes} are data structures designed to support search.
The performance of an index can be measured by:
\begin{itemize}
    \item time
    \item space
    \item storage
    \item latency
    \item throughput
\end{itemize}
An inverted index is a list for each word a list of documents
where the word appears. It's essentially the inverse of a
classical index (who would have thought?).

\textbf{Inverted index} is composed of multiple composed data structures:
\begin{itemize}
    \item Dictionary: a list of all the terms in the collection.
    \item \textbf{Document IDs}: a list of all the documents in the collection
    associated to a positive integer.
    \item \textbf{Term IDs}: a list of all the terms in the dictionary
    associated to a positive integer.
    \item \textbf{Posting lists}: each term is associated with a list of documents 
    where the term appears and other information. The list is sorted
    by document ID. These lists are stored sequentially in a file and we store the
    offset of the start of the list for each term.
    If we have different information for each term-document pair it's better 
    to store the information in separate files.
    \item \textbf{Lexicon}: the string representation and some statistics of the terms.
    \item \textbf{Document index}: contains information on the single documents like
    their length. The actual contents of the document are not needed until the very
    last step of the retrieval process.
    \item Collection statistics: contains information on the whole collection.
    \item Direct index: an index mapping documents to terms and their frequencies.
    This is not frequently used.
\end{itemize}
\textbf{Bold} items are the mandatory ones.

With a simple posting list no ranking is possible, we need to
add some information to the list. We can add the term frequency
in the document, the position of the term etc. We can use
these additional information to rank the documents.
Frequencies are actually counts not real frequencies.

What is a "term" for the purpose of indexing?
Any string between two separating symbols. In this case we are
referring to \textbf{full-text indexing}.
Another approach is \textbf{key-word indexing} where we index
only the most important words in a document. These words are
extracted from a list.
\textbf{Key-word indexing} allows for a more compact index because
there will be less \textbf{posting lists} and also they will be shorter.
We can also use \textbf{phrase indexing} where we index the whole
phrase instead of single words. In this case we use a list of 
key phrases.
We can also use \textbf{n-gram indexing} where we index some n-length
word sequences. This makes the index and lexicon significantly longer
but \textbf{posting lists} are shorter.

What do we put in a posting?
The minimum is the document ID. We can also put the term frequency (count),
the impact score (how important the term is in the document) or even
a list of positions of the term in the document.

NEVER store document IDs and frequencies in the same file.