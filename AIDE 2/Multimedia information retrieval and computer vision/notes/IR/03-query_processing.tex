\subsection{Query processing}
\label{sec:query_processing}

We can process query terms in two ways:
\begin{itemize}
    \item Conjunctive (AND)
    \item Disjunctive (OR)
\end{itemize}

After processing the query we select what documents to return
and in what order.
\begin{itemize}
    \item \textbf{Boolean retrieval}: returns documents that match the query.
    \item \textbf{Ranked retrieval}: for each document we compute a score
    an order the documents by score.
    We assume that if a term does not appear in the query it gives 0 score.
\end{itemize}

Boolean retrieval can be useful only for expert users, they know
what they are looking for and are able to consume all the results.
It can be used also for applications that can easily consume all the results.

For the general public we need ranked retrieval. In this case the query
is written in "free text queries", these queries are more similar to
natural language.

We want to order the results in such a way that the ones that are more
likely to be useful are ordered first.
The score function is in the form: $s: Q\times D \rightarrow \mathbb{R}$,
where $Q$ is the set of queries and $D$ is the set of documents.

\subsubsection{Term at a time}
\label{sec:term_at_a_time}

For each term that appears in the query we scan the posting list and
compute the score for each document. We save the score in a table called
\textbf{accumulator}.

\textbf{Accumulators} are cache friendly and can be easily used to find
the top $k$ documents.

This technique is not the best with boolean queries.

\subsubsection{Document at a time}
\label{sec:document_at_a_time}

We open an iterator overy every posting list. We scan the posting lists
in parallel and because the posting lists are sorted. We compute the 
\textbf{final score} for a single document ignoring the other documents
and then we move to the next document.
In this way we can stop scanning once we found the top $k$ documents.

This method is less cache friendly than the term at a time method but
is more widely used because it is more efficient.

\subsubsection{Distributed query processing}
\label{sec:distributed_query_processing}

All queries are sent to a manager machine.
The manager then sends messages to many index servers.
Each index server does some portion of the query processing.
The manager organises the results and returns them to the user.

\subsubsection{Distributed index}
\label{sec:distributed_index}

We can distribute the index by document or by term.
Usually we distribute by document and build for every partition
its own index.

All indices return the top $k$ documents to the manager.
The manager then selects the top $k$ documents from all the results.

To distribute the system we need also \textit{global} statistics that needs to be
stored somewhere, the local statistics are not enough.

If we distribute by term the manager has much more to do but we need less computers.
Also load balancing can be an issue, because a term can be more popular than another.

\subsubsection{Caching}
\label{sec:caching}

It's important to store a software cache. Query caching can be refreshed
only once every some months, while term caching has to be refreshed
more often.

\subsubsection{Compression}
\label{sec:compression}

Compression is important because it allows to store more data in memory.
Ideally we want to keep the whole vocabulary in memory and also some posting lists.

The Heap's law says that the size of the vocabulary $M=|V|$ is proportional to
$M=kT^b$ where $T$ is the number of tokens.

The Zipf's law says that the frequency of a term is inversely proportional to its rank.
The rank of a term is calculated by sorting the terms by frequency. $cf_i=\alpha/i$ where
$i$ is the rank.

\subsubsection{Scoring functions}
\label{sec:scoring_functions}

A common measure is the Jaccard coefficient:
\[
    J(A,B) = \frac{|A\cap B|}{|A\cup B|}
\]

This measure is used to evaluate the overlap between two sets.
It's always in $[0,1]$.

This measure does not take into account 
how many times the term appears in the document.
It does not take into account the rarity of the term, 
the occurrences and the size of the document.

We can represent documents by \textbf{bag of words} vectors model.
For each combination of term and document we store the frequency of the term in the document.

\textit{Hans P. Luhn} proposed that the weight of a term should be proportional to
its absolute frequency in the document.

The term-frequency weight is defined as:
\[
    w_{t,d} = 
    \begin{cases} 
        1 + \log(tf_{t,d}) & \text{if } tf_{t,d} > 0\\
        0 & \text{otherwise}
    \end{cases}
\]

We should also keep in account the absolute frequency of a word in the collection, because
more common words are less informative.
As invented by \textit{Karen Sp\"arck Jones} we should also use the inverse document frequency.
\[
    w_{t,d} = 
    \begin{cases}
    (1 + \log(tf_{t,d})) \cdot \log\left(\frac{N}{n_t}\right) & \text{if } tf_{t,d} > 0\\
    0 & \text{otherwise}
    \end{cases}
\]

Where $N$ is the number of documents and $n_t$ is the number of
documents that contain the term $t$.

So the score of a document is $\sum_{t\in q} w_{t,d}$.

We can store the weights in the index.

To compute the score of a document given a query, we can represent
the query as a count BoW, and then compute the dot product between
the query vector and the document vector.

\textbf{Cosine similarity} is a better measure than the dot product, we have to
normalize the vectors and then compute the dot product.
