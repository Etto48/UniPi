\subsection{Relevance feedback}
\label{sec:relevance_feedback}

The information need is the cause of the query
that the user issues to the IR system.
This need can be categorised with a variety of
dimensions: number of relevant documents, type 
of information needed, the type of the task
that led to the need.
A query can represent different information needs
that may require different algorithms to produce
the best rankings.
A query can be a poor representation of the
information need. Users may found difficult to
expresses the information need, they may be
encouraged to issue short queries, also the
same query string may represent different
information needs.

It's usually difficult to formulate a good query
without knowing the collection and the retrieval
environment. Usually the user has to issue a
query, examine the results and then reformulate
the query. IR is usually interactive and iterative.

Interaction with the system occurs during the
formulation and reformulation of the query and 
during the examination of the results.
Users can't change the ranking algorithm but can 
change the results with their interaction.

We can represent both the query and the documents as
vectors. Relevant documents have similar vectors to
the query. To make the vectors more similar we can
modify the weights of the terms in the query,
usually positive feedback is used to increase the
weights of the terms in the query that are present
in relevant documents. Sometimes negative
feedback is used to decrease the weights of the
terms in the query that are present in non-relevant
documents. We can also remove the terms that are
present only in non-relevant documents.

The optimal query maximizes the difference between
the average vector representing the relevant
documents and the average vector representing the
non-relevant documents. The Rocchio algorithm
modifies the query vector by adding a linear
combination of the vectors representing the
relevant and non-relevant documents.
\[
    \vec{q'}=\alpha\vec{q}+\beta\frac{1}{|D^+|}\sum_{\vec{d}\in D^+}\vec{d}
        -\gamma\frac{1}{|D^-|}\sum_{\vec{d}\in D^-}\vec{d}
\]

Typically $\alpha=8$, $\beta=18$ and $\gamma=4$.
Query terms with negative weights are removed.
Typically only the top $k$ terms with highest weights
in the relevant documents are added to the query.

The relevance feedback generally improves the performance
(recall and precision) of the IR system. It's most useful for
increasing recall in situations where the recall is important.

Positive feedback is more valuable than negative feedback.
That's why $\gamma<\beta$. Many systems set $\gamma=0$.

Users in general are reluctant to provide explicit feedback.

\textbf{Pseudo-relevance feedback} is a technique that
tries to estimate what document the users would have
marked as relevant. We assume that the top $m$ documents
are relevant and use them to reformulate the query.

In the expanded query the terms are not re-weighted, therefore
we need to re-weight the query terms, including the expanded ones.

This algorithm suffers form \textbf{query drift}: this problem
arises when documents used for RF contain few or no relevant
documents. The algorithm will add terms that are poor
at detecting relevance and the performance will decrease.

\textbf{Pseudo-relevance feedback} is a technique that
will improve some queries but also harm others.

We can expand the query also with other techniques, like
the \textbf{thesaurus-based expansion} that performs expansion
by adding synonyms of the terms in the query.
We can weight the new terms with less weight than the
original terms.
This technique generally increases recall but can
decrease precision.
