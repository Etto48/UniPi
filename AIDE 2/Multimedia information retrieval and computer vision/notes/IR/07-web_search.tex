\subsection{Web search}
\label{sec:web_search}

The types of queries that are issued to a web search engine
are:
\begin{itemize}
    \item \textbf{Navigational queries}: to reach a particular site
    that the user has in mind. They are the 20\% of the queries.
    \item \textbf{Informational queries}: to find information 
    assumed to be present on one or more webpages. They are
    the 48\% of the queries.
    \item \textbf{Transactional queries}: to reach a site where
    further interaction will happen. They are the 30\% of the
    queries.
\end{itemize}

Google has at least 30 trillion pages in the indexes of its search
engine.

What's relevant in a web search depends also on time. The meaning
of a query can change over time, take for example "president of
the united states" or "world cup".

Query volume and type also changes in different locations, the
meaning of a query can change also depending on the location.

The search engine has to personalise the query depending on the
user's context. Users have different interests, which are
reflected in their short and long-term search history.
Queries cold be ambiguios for the search engine, 
personalisation signals help to resolve that.

In a web search engine, the query also has the user context
in it.
So the goal is to estimate $P(r|d,x)$, where $r$ is the relevance
of the document $d$ to the query and context $x$.

We need to find features that are correlated with relevance.
Most of the web is junk so we need to isolate the relevant
information.

In web search we have hyperlinks, these can be used to extract
information about a page.
The anchor text of a hyperlink is a good indicator of the
content of the page it points to.
The number of hyperlinks to a page is also a good indicator
of the importance of the page.
A hyperlink is a transition of authority on a certain topic from
a page to another.

The first search engines used the number of hyperlinks to a page
to rank the page, this does not depend on the query but only
on the page. This system is susceptible to link spamming.
The \textbf{PageRank} algorithm simulates a very large number 
of users that navigates the web by following hyperlinks.
At each time step, the user jumps to a random page with
a probability $1/d$ where $d$ is the number of pages linked
from the current page.

To compute the PageRank we first need the adjacency matrix.
We transform the adjacency matrix into a transition matrix
by dividing each row by the sum of its elements.
Then we transpose the matrix and we get a Markov chain.
Each element of the matrix is $P(j|i)$ (the probability of
moving to page $j$ being in page $i$).
THe Markov chain is ergodic so there is no recurrent visit
pattern.
For any ergodic Markov chain there is a unique long-term
proability distribution.
To find this distribution we need to find $Mx=x$ (eigenvector)
where $M$ is the Markov chain matrix and $x$ is the probability
distribution.
To avoid dead ends, PageRank introduces the teleportation
computation: at each time step the user can jump with
small probability to a random page.
To get the Google matrix $G$ we have to add the teleportation
computation to the Markov chain matrix.
\[
    G = (1-\lambda)M + \lambda 1/n
\]
Where $n$ is the number of pages.
We can prove that the Google matrix is an ergodic Markov chain.
Nowadays the PageRank is not taken into account (too much) by
Google, but it's still a good indicator of the importance of
a page.

To get the data for the PageRank we need to crawl the web.
A crawler is a program that exploits the connected structure
of the web to find new pages (We assume the web to be connected).

A crawler has a priority queue called frontier.
We start from a set of well-connected pages (seed pages).
We consume the frontier by fetching a page, parsing it and
extracting the links. We add the links to the frontier.
The problem is that the frontier can grow very fast.
The process must be resistant to errors and run on many machines.
How many times we should re-crawl a page? We have to balance 
freshness and politeness. We have to skip duplicates and
near-duplicates.
We have 3 sets of pages: 
\begin{itemize}
    \item \textbf{Downloaded}: the pages that we have downloaded.
    \item \textbf{Discovered}: the pages that were discovered from
    the downloaded pages.
    \item \textbf{Undiscovered}: the pages that are not yet
    discovered by the crawler.
\end{itemize}

The frontier has to be sorted by a priority function.
We have also to implement an url filter to avoid crawling
the same page multiple times.
We need a content filter to avoid duplicates and near-duplicates.
Nutch is a popular distributed open-source crawler.

To make the web search feasible we have to distribute it.
We can distribute the index by term or by document.
With the term distribution we have to use multiple servers
for a single query.
With the document distribution each server is responsible
for a subset of the documents.
We use map-reduce to process the query and merge the results.
