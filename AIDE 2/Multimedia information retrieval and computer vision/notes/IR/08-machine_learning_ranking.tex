\subsection{Machine learning for ranking}
\label{sec:machine_learning_ranking}

Can we use machine learning to rank documents? (yes)
The simplest model is a linear regressor.
Actually machine learning for ranking works really good.
In 2008 Google was already using 200 features to rank documents.

Machine learning models are basically one of two kinds:
\begin{itemize}
    \item \textbf{Generative models}: they generate data.
    They work well with low number of training examples.
    \item \textbf{Discriminative models}: they discriminate between
    classes. They work better with a lot of training examples.
\end{itemize}

Regression is not a good model for ranking, because we do not
care about the value of the score, but only of the order of the
documents.

We want specifically designed ML techniques for ranking documents
w.r.t. a query (learning to rank).

To train such models we will of course need training data.
Also clickthrough data can be really helpful to generate more
training data.

Whatever model we use, it will be slower than the simple
BM25 algorithm.

We cannot process the whole dataset, so we build what's called
a cascade architecture. Actually BM25 is already a cascade
architecture: we first filter the documents with simple
boolean inclusive retrieval and then we rank the documents
with BM25.

We can re-rank the top $k$ documents that are the output of
BM25 with a machine learning model, of course the old score
can be a feature of the model.
Each step of the cascade can return less and less documents,
so we can use more complex models.

It's important that every step of the cascade is effective
because if we remove important documents at the first step
the second step cannot fix that. So we focus on recall-based
metrics to avoid losing good documents even if they are in a
suboptimal order.

To train a ML model we need a training set of queries and
documents with relevance judgements.
For each document we need a feature vector, and also
the query must be represented as a feature vector.

We have multiple features to feed into the model, they can be:
\begin{itemize}
    \item \textbf{Query-document features}: they depend on the
    query-document pair.
    \item \textbf{Query features}: they depend only on the query.
    \item \textbf{Document features}: they depend only on the
    document.
\end{itemize}

To train the model we need to:
\begin{itemize}
    \item Sample the top documents from the BM25 output.
    \item Compute the features
    \item Train the model
\end{itemize}

During inference we need to:
\begin{itemize}
    \item Sample the top documents from the BM25 output.
    \item Compute the features
    \item Apply the model and re-rank the documents
\end{itemize}

We will need a training-set and also a test-set
(optionally also a validation-set).

We need a loss function to train the model.
We have different types of loss functions:
\begin{itemize}
    \item \textbf{Point-wise loss}: evaluate one single output.
    \item \textbf{Pair-wise loss}: evaluate pairs of outputs.
    \item \textbf{List-wise loss}: evaluate a whole list of
    outputs.
\end{itemize}

The models that we can use are:
\begin{itemize}
    \item \textbf{Linear models}: they are simple and only use
    a vector of weights. They need the gradient of the loss
    function.
    \item \textbf{Tree-based models}: a random forest is a good
    model to create contributions that will be summed up to
    get the final score. We can use LambdaMART and gradient-boosted
    regression trees.
\end{itemize}

We can also use bags of models (ensamble methods) to get better
results.