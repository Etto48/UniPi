\subsection{Evaluation}
\label{sec:evaluation}

There are two aspects to measure about an IR system: effectiveness and efficiency.
Efficiency is easy to measure, it is the time it takes to perform a query.
Effectiveness is harder to measure, it is related to user satisfaction.
We need a way to measure the relevance of the results of a query for
a user.
There are three elements to consider in the evaluation of an IR system:
\begin{itemize}
    \item \textbf{Benchmark collection} collection.
    \item \textbf{Benchmark queries} collection.
    \item \textbf{Assessment} of relevance for each query and document
    For each query-document pair we need to know if the document is relevant
    to the query or not. (qrels)
\end{itemize}

The metric will have a maximum value of 1, that represents a perfect system and
cannot be reached in practice.

Relevance assessment can be done in two ways:
\begin{itemize}
    \item \textbf{Binary assessment} the document is either relevant or not.
    \item \textbf{Graded assessment} the document is assigned a score.
    Graded assessment is more informative but also more difficult to obtain.
\end{itemize}

Assessments are extremely expensive to obtain, one possible solution is crowdsourcing,
this can lead to noisy data because the assessors are not trained.

Relevance assessments are the most important part of the IR system development process
because if we don't know how good the system performs we cannot improve it.

To create a benchmark query collection we can use a query log, this is easier to obtain
with web search engines.

When query logs cannot be used we need an expert to create the queries.

In order to make experimentation sound we need the benchmark and assessments to be
publicly available.

There exists several benchmarks for IR systems, the Text REtrieval Conference (TREC)
makes them available.

There is a tool called \texttt{trec\_eval} that is the only tool that should be used
to evaluate IR systems.

To evaluate an IR system we take the \textbf{topic} as input, create a \textbf{run}
where we get the top-k documents for each topic and then we get the \textbf{run}
assessed (with the qrels) and we use this to get the evaluation metric.

A \textbf{topic} is composed of
\begin{itemize}
    \item \textbf{Title}: a brief description of the topic, can be used as a query.
    \item \textbf{Description}: a more detailed description of the topic.
    \item \textbf{Narrative}: a description of the relevant documents used to asses the
    relevance of the results.
\end{itemize}

Usually an assessment is done by pooling:
\begin{itemize}
    \item We select some systems to run the queries.
    \item We get the top-k ($50<k<200$) documents for each query.
    \item We merge the results of the systems.
    \item We remove the duplicates.
    \item We present the documents in random order to the assessors.
\end{itemize}

When creating a \textbf{run} we need to save both the rank and the score of the document,
the score is used to evaluate how the system handles ties.

\subsubsection{Set based metrics}
\label{sec:set_based_metrics}

With non ranked results we can use set-based metrics like precision, 
recall and F1-score.
In a expert search we want a high recall, precision is not as important.

We can compute the average with arithmetic mean or geometric mean.

With ratios with the same numerator it's better to use the geometric mean.
With ratios with the same denominator it's better to use the arithmetic mean.

\subsubsection{Rank based metrics}
\label{sec:rank_based_metrics}

With ranked results it's better to use rank based metrics.
The most common rank based metrics are:
\begin{itemize}
    \item \textbf{Precision at k} $P@k$ is the proportion of relevant documents in the top-k.
    With lower k we usually get higher precision.
    \item \textbf{Recall at k} $R@k$ is the proportion of relevant documents in the top-k.
    With higher k we get lower recall.
    \item \textbf{Mean Average Precision} $MAP$ is the average of the precision 
    at each relevant document.
\end{itemize}

We can compute the Precision-Recall curve by plotting the precision at each recall level.
We can replace the curves with the maximum seen so far to get the interpolated precision 
and recall. This representation is difficult to interpret.

To compute the $MAP$ we compute the rank of each relevant document $K_i$ and then we compute
Whenever we encounter a relevant document we compute the precision at that point $P@K_i$ and
then we compute the average of the precision at each relevant document. Usually we compute
the $MAP$ at cutoff 1000. (So technically it's $MAP@1000$)

It's easy to prove that $MAP$ is equivalent to the area under the Precision-Recall curve.

If we want to use graded relevance assessment we can use the discounted cumulative gain ($DCG$).
\[
    DCG@k = 
    \begin{cases}
        \sum_{i=1}^k r_i & \text{if } k < b \\
        DCG@b(k-1)+r_k/\log_b(k) & \text{if } k \geq b
    \end{cases}
    = \sum_{i=1}^k\frac{r_i}{max(1,\log_bi)}
\]

In this formula $b=2$ is an impatient user, $b=10$ is a patient user.
Usually we use $b=2$.

Unfortunately $DCG$ is not bounded, it goes from 0 to $\infty$.

This metric introduced a \textbf{user model}. Any rank based evaluation metric assumes
a user model.

All models are wrong, but some are useful.

To solve the problem of the unboundedness of $DCG$ we can use the normalized discounted
cumulative gain ($nDCG$).
\[
    nDCG@k = \frac{DCG@k}{IDCG@k}
\]

Where $IDCG@k$ is the ideal discounted cumulative gain at cutoff $k$. This is the
highest possible $DCG$ that can be obtained for a given query.
We can compute the $IDCG$ by sorting the documents by relevance and then computing the $DCG$.

For our systems we can expect $nDCG$ to be between 0.1 and 0.2.

Another metric is the rank-biased precision ($RBP$).
In this user model, the user will continue looking at the results with a probability $p$.
Usually we use $p=0.5$ for impatient users.

\[
    RBP@k = (1-p)\sum_{i=1}^kp^{i-1}r_i=(1-p)\sum_{k\in R}p^{k-1}
\]

Another metric that is not very useful but works well with a single relevant document is
the mean reciprocal rank ($MRR$).

\[
    MRR = max_{i\in R} \frac{1}{i}
\]

The max is used only if we have multiple relevant documents, otherwise it's only
$\frac{1}{i}$.

This metric is very sensitive to variations in the top ranks.

\subsubsection{Statistical evaluation}
\label{sec:statistical_evaluation}

In theory we should use not a sample of queries but all the queries possible
on a finite and predefined set of documents. Because Q (the set of queries) is
a random variable we have to compute the expected value of the metric.

In practice this is not feasible because the number of queries is too large
and we do not have the ground truth for all the queries and documents.
We use a subset of the queries and compute the average metric.

Once we have the average metric, can we conclude that a system is better than
another system?
Obviously not, the only way to do this would be with the ideal set of all 
possible queries.
To assert that a system is better than another we need to use statistical tests.
This tests can tell us if the difference between two systems is significant or not
and with what confidence.

The most common statistical test is the t-test.
Another test that is simpler is the Wilcoxon signed-rank test.

You should fix the p-value before running the test, usually it's 0.05.