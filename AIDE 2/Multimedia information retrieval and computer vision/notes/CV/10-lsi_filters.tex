\subsection{Convolution}
\label{sec:convolution}

When we compute the output of an LSI filter we will need to compute
the convolution of the input image with the filter.
Because the convolution is written as
\[
    f[x,y]*g[x,y]=\sum_{i=-\infty}^{+\infty}\sum_{j=-\infty}^{+\infty}f[i,j]g[x-i,y-j]
\]
the convolution filter needs to be flipped before applying it to the input image.
We can denote the elementwise multiplication of two matrices as $\odot$.

Usually we write an image as an infinite matrix to simplify the notation.
In practice the actual image is limited in a finite box of $I_h\times I_w$ pixels
and the values outside this box are set to zero.

When we actually want to compute the convolution of an image
what happens on the pixels on the border of the image?
While we the whole filter stays inside the image we do not have any problem.
If the filter is centered on a border pixel (or is in any ways partially outside the image)
we start to have problems.

If we drop the values obtained with the filter outside the image with a filter of 
size $k\times k$ we will have an output image of size 
$(I_k-\lfloor\frac{k}{2}\rfloor)\times(I_w-\lfloor\frac{k}{2}\rfloor)$.
So for even the smallest filters of size $3\times 3$ we will lose a border of $1$ pixel.

To avoid losing a lot of pixels we can use a technique called \textbf{padding}.
Padding can be:
\begin{itemize}
    \item \textbf{Constant padding}: we add a border of constant values to the image.
    This solution downgrades the quality of the border pixels.
    \item \textbf{Extended padding}: we use the same color of the border pixel to fill the border.
    This is not a good solution for large filters.
    \item \textbf{Mirror padding}: we mirror the image to fill the border.
    This solves problems with large filters.
\end{itemize}

One important class of filters are the so-called \textbf{separable filters}.
A filter $h[x,y]\in\mathbb{Z}^{n\times m}$ is separable if it can be written as:
\[
    h[x,y]=h_x[x]*h_y[y]
\]
where $h_x[x]\in\mathbb{Z}^n$ and $h_y[y]\in\mathbb{Z}^m$.
We can write it also as an outer product:
\[
    h[x,y]=(h_x h_y^T)[x,y]
\]
A very important result is that if a filter is separable it can
be applied as two 1D convolutions instead of a 2D convolution.
The computational complexity of a generic 2D convolution is $O(nm)$
while the complexity of two 1D convolutions is $O(n+m)$.
For big filters this is a significant speedup.

\subsubsection{Gradient}
\label{sec:gradient}

We can write (for the taylor expansion):
\[
    \begin{cases}
        f[x-h]=f[x]-hf_x[x] \\
        f[x+h]=f[x]+hf_x[x]
    \end{cases}
\]
We set $h=1$ because it's the smallest interval in $\mathbb{Z}$.
We can calculate the derivative as:
\[
    \begin{cases}
        f[x]-f[x-1]=f_x[x] \\
        f[x+1]-f[x]=f_x[x]
    \end{cases} \\
\]
So the central derivative is:
\[
    f_x[x]=1/2(f[x+1]-f[x-1])
\]
the forward derivative is:
\[
    f_x[x]=f[x+1]-f[x]
\]
and the backward derivative is:
\[
    f_x[x]=f[x]-f[x-1]
\]

This is the same as the LSI filter
(remember to flip the filter)
for the central derivative:
\[
    h_x[x]=\begin{bmatrix}
        1/2 & 0 & -1/2
    \end{bmatrix}
\]
for the forward derivative:
\[
    h_x[x]=\begin{bmatrix}
        1 & -1 & 0
    \end{bmatrix}
\]
and for the backward derivative:
\[
    h_x[x]=\begin{bmatrix}
        0 & 1 & -1
    \end{bmatrix}
\]

So the gradient can be computed as:
\[
    \nabla f[x,y]=f[x,y]*h_x[x]*h_y[y]
\]

If we want to compute the magnitude of the gradient
we lose linearity because we have to compute the square root 
of the sum of the squares:
\[
    \nabla f[x,y]=\sqrt{f_x[x,y]^2+f_y[x,y]^2}
\]

\subsubsection{Laplacian}
\label{sec:laplacian}

The discrete Laplacian operator is defined as:
\[
    \nabla^2 f[x,y]=f_{xx}[x,y]+f_{yy}[x,y]
\]

The second derivative can be computed as:
\[
    f_{xx}[x,y]=\begin{bmatrix}
        1 & -2 & 1
    \end{bmatrix} * f[x,y]
\]
and
\[
    f_{yy}[x,y]=\begin{bmatrix}
        1 \\
        -2 \\
        1
    \end{bmatrix} * f[x,y]
\]

If we want to compute the Laplacian we can use the Laplacian filter:
\[
    L[x,y]=\begin{bmatrix}
        1 & -2 & 1 \\
    \end{bmatrix} * \begin{bmatrix}
        1 \\
        -2 \\
        1
    \end{bmatrix} = \begin{bmatrix}
        0 & 1 & 0 \\
        1 & -4 & 1 \\
        0 & 1 & 0
    \end{bmatrix}
\]
This filter is important because
\[
    \nabla^2 f[x,y]=L[x,y]*f[x,y]
\]

\subsubsection{Gaussian filter}
\label{sec:gaussian_filter}

The Gaussian filter in two dimensions is defined as:
\[
    g[x,y,\sigma]=ce^{-\frac{x^2+y^2}{2\sigma^2}}
\]
Where $c$ is a normalization constant and $\sigma$ is called spread.
We choose $c$ so that the sum of the filter is $1$.
If the sum of all the elements of the filter is $1$ it's called a \textbf{smoothing filter}.
If the sum of all the elements of the filter is $0$ it's called a \textbf{differentiating filter}.

The Gaussian filter is \textbf{rotation-invariant} so nothing changes if we flip the filter.
The maximal contribution of this filter is always in the center.

How do we design a Gaussian filter?
We chose a size $k\times k$ and a spread $\sigma$.
For the moment we ignore the constant $c$.
We compute the filter as:
\[
    \hat{g}[x,y,\sigma]=e^{-\frac{x^2+y^2}{2\sigma^2}}
\]
for each $x,y$ in the size of the filter.
We compute the normalization constant as:
\[
    c=\frac{1}{\sum_{x,y}\hat{g}[x,y,\sigma]}
\]

If we want to have a filter with all integer values we have to rescale $\hat{g}$.
We multiply all the values by:
\[
    \frac{1}{\min_{x,y}\hat{g}[x,y,\sigma]}
\]
For a $3\times 3$ filter with $\sigma=0.85$ we have:
\[
    g[x,y]=\frac{1}{16}\begin{bmatrix}
        1 & 2 & 1 \\
        2 & 4 & 2 \\
        1 & 2 & 1
    \end{bmatrix}
\]
We can compute the output of the filter as:
\[
    f[x,y]*g[x,y]=
    c\sum_{i=-\infty}^{+\infty}\sum_{j=-\infty}^{+\infty}f[i,j]
    e^{-\frac{i^2+j^2}{2\sigma^2}}
\]
We can factor out half the exponential because the indices are independent:
\[
    f[x,y]*g[x,y]=
    c\sum_{i=-\infty}^{+\infty}(e^{-\frac{i^2}{2\sigma^2}}\sum_{j=-\infty}^{+\infty}f[i,j]e^{-\frac{j^2}{2\sigma^2}})
\]
We can rewrite the sum as a convolution:
\[
    f[x,y]*g[x,y]=
    c*g_x[x]*g_y[y]*f[x,y]
\]
This means that the Gaussian filter is separable.

~ "Can't you fall in love with the Gaussian filter?"
