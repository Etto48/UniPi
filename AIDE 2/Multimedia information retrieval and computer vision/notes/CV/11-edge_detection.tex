\subsection{Edge detection}
\label{sec:edge_detection}

An edge is a region in an image where the intensity changes abruptly.
A point in an image is an edge point if points in its neighborhood are
significantly different.
"Change" makes us think immediately to derivatives.

An edge detection algorithm given an image will output a list of edge points.

A simple edge detection algorithm is a threshold on the gradient.
The gradient is a vector so we need to compute its magnitude.
The threshold is needed in order to decide what change is small and what change is big.

In theory this approach works well but in practice it produces too many edge points and also
needs a parameter.

With the second derivative, we can detect zero-crossings.
A good way to compute the second order derivative is the Laplacian operator.
We search for $\nabla^2 f[x,y]=0$.

In the real world we have noise, in this case the derivative will produce
a lot of zero-crossings.

A moving average filter can be used to smooth the image, but this remove high frequency
components, including edges.

The solution is to smooth the image with a Gaussian filter.
This filter is better at preserving edges.
To do the derivative of a convolution we have to compute $(f[x]*g[x])_x$
we can rewrite this as $f[x]*g_x[x]$.
The derivative of the Gaussian filter is a new filter that computes everything we need in one step.

If we compute the derivative of a Gaussian filter we obtain
\[
    \nabla g[x,y,\sigma]=-\frac{1}{2\pi\sigma^4}e^{-\frac{x^2+y^2}{2\sigma^2}}\begin{bmatrix}
        x \\
        y
    \end{bmatrix}
\]
This two filters are both separable. Because the gaussian is rotation-invariant, we only have
to compute the two filters in one dimension.

We still have the problem that the gradient is two-dimensional.
For this reason we will need the Laplacian.
We compute the Laplacian of the Gaussian filter for the differentiation property. 
So we get that:
\[
    \nabla^2 (f[x,y]*g[x,y,\sigma]) = \nabla^2g[x,y,\sigma]*f[x,y] =
        L[x,y]*g[x,y,\sigma]*f[x,y]
\]
We can easily compute the Laplacian of the Gaussian filter an apply it to the image
because both the filters are separable and rotation-invariant.
$L[x,y]*g[x,y,\sigma]$ is called \textbf{Laplacian of Gaussian (LoG)} filter.