\subsection{Computer vision basics}
\label{sec:computer_vision}

Let's start with some definitions:
\begin{itemize}
    \item \textbf{Analog image}: $f(x,y):\mathbb{R}^2 \rightarrow \mathbb{R}$.
    \item \textbf{Digital image}: $f[x,y]:\mathbb{Z}^2 \rightarrow \mathbb{Z}$.
    \item \textbf{Pixel}: Picture elements (the argument of the previous function).
    \item \textbf{Intensity value}: The intensity value of a pixel 
    (the output of a \textbf{digital image})
    is represented by an integer in the range $[0,2^n-1]$, this is called
    $n$-bit color depth. Usually it's $n=8$.
    \item \textbf{Multispectral image}: $f[x,y]: \mathbb{Z}^2 \rightarrow \mathbb{Z}^d$ with $d>1$.
    \item \textbf{Color image}: Multispectral image with $d=3$. Usually every channel
    has an intensity value over $n$ bits. The standard it's $n=8$ so a total of 24 bits.
    \item \textbf{Filter}: image manipulation is carried out thanks to filters. A filter
    (aka system) is a black box that takes as input an image and gives as output another image.
    A filter is uniquely identified by its \textbf{filter operator} $S$. A filter is a mapping
    between all possible input images $f$ and the corresponding output images $g$:
    $g[x,y]=S(f[x,y])$.
\end{itemize}

A very important filter is the \textbf{$3\times 3$ moving average} filter. It's defined as:
\[
    g[x,y]=1/9\sum_{i=-1}^{+1}\sum_{j=-1}^{+1}f[x+i,y+j]
\]
This filter givs as output that for each pixel has the average of the $3\times 3$ neighborhood
in the input image. This effect is called \textbf{smoothing}.

Filters has many properties:
\begin{itemize}
    \item \textbf{Additivity}: $S(f+g) = S(f) + S(g)$.
    \item \textbf{Homogeneity}: $S(\alpha f) = \alpha S(f)$ with $\alpha\in\mathbb{R}$.
    \item \textbf{Superposition}: $S(\alpha f + \beta g) = \alpha S(f) + \beta S(g)$. This
    is called a \textbf{linear filter}.
    \item \textbf{Causality}: $\forall x\le x_0,y\le y_0, f[x,y]=0 \Rightarrow S(f[x,y])=0$.
    \item \textbf{Shift invariance}: $g[x-x_0,y-y_0]=S(f[x-x_0,y-y_0])$. A filter
    that is linear and shift invariant is called a \textbf{linear shift-invariant (LSI) filter}.
\end{itemize}

Our vision system (eyes) can be approximated with a LSI filter.

We can define our \textbf{impulse function} as: 
\[
    \delta: \mathbb{Z}^2 \rightarrow \mathbb{Z} \\
    \delta[x,y]=\begin{cases}
        1 & \text{if } x=y=0 \\
        0 & \text{otherwise}
    \end{cases}
\]

For a LSI filter we have that:
\[
    S(\delta[x,y])=h[x,y]
\]

For the $3\times 3$ moving average filter we have that:
\[
    h[x,y]=\begin{cases}
        1/9 & \text{if } x,y\in\{-1,0,1\} \\
        0 & \text{otherwise}
    \end{cases}
\]

For any LSI filter we can write $h[x,y]$ as a sum of scaled and shifted impulse functions:
\[
    h[x,y]=\sum_{i=-\infty}^{+\infty}\sum_{j=-\infty}^{+\infty}h[i,j]\delta[x-i,y-j]
\]
For a LSI filter we have that:
\[
    S(f[x,y])=\sum_{i=-\infty}^{+\infty}\sum_{j=-\infty}^{+\infty}h[i,j]f[x-i,y-j]
\]

We define the \textbf{convolution} of two functions $f$ and $g$ as:
\[
    f[x]*g[x]=\sum_{i=-\infty}^{+\infty}f[i]g[x-i] \\
    f[x,y]*g[x,y]=\sum_{i=-\infty}^{+\infty}\sum_{j=-\infty}^{+\infty}f[i,j]g[x-i,y-j]
\]
So we have that any LSI can be written as:
\[
    S(f[x,y])=h[x,y]*f[x,y]
\]
In actual applications $h[x,y]$ will be designed by hand
or learned from data.
